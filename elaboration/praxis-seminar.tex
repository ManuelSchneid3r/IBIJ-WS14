\documentclass[fleqn,envcountsame,runningheads,10pt,a4paper]{llncs}
\usepackage[utf8]{inputenc}
\usepackage{enumerate} 
\usepackage{verbatim}
\usepackage{graphicx}
\usepackage{amsmath}
\usepackage{amsfonts}
\usepackage{amssymb}
\usepackage{cite}
\usepackage{url}
\usepackage{ngerman}
\usepackage{geometry}
\usepackage{hyperref}
\geometry{a4paper,left=30mm,right=30mm,top=35mm,bottom=25mm}

\setlength\parindent{0pt}%Festlegen des Absatzeinzuges
\setlength\mathindent{0pt}%Festlegen des Einzuges für abgesetzte Formeln

\pagestyle{headings}

\begin{document}
%==============================================================================
\title{Per-Circuit TCP-over-IPsec Transport for Anonymous Communication Overlay Networks} 
\titlerunning{}
\author{Manuel Schneider\\ms476@pluto.uni-freiburg.de}
\authorrunning{}
\institute{Seminar am Lehrstuhl für Kommunikationssysteme \\ Albert-Ludwigs-Universität Freiburg}
\maketitle
%==============================================================================
%\begin{abstract}
%Das Ziel bei der Durchführung des Praxis-Proseminars ist, dass sich die Teilnehmer mit (erweiterten) %Fragestellungen im Bereich Kommunikationssysteme beschäftigen. Der Fokus liegt auf einem %praktischen Zugang zur Thematik.
%\end{abstract}

%==============================================================================
\section{Einleitung}
\label{sec:intro}

Problem Anonymität in Internet\\
Lösung: Anonymsierungsnetzwerke\\
Beispiel Tor\\
Nachteile Tor (Peformance)\\
Scherpunkt, Thema dieser Ausarbeitung\\
Problemeingrenzung (Durchsatz, Latency, woher kommts? Refernez auf sec:tor ) \\
Einleitung der behadelten Lösungsansätze (TCPoDTLS, PCTCP)\\
Was ist das Ziel vom PCTCP? \\



%==============================================================================
\section{Grundlagen}
\label{sec:basics}

%------------------------------------------------------------------------------
\subsection{TOR}
\label{sec:tor}

Tor ist ein Netzwerk zur Anonymisierung der Verbingungsdaten. Es ist weit verbreitet und wird ständig weiter entwickelt. Das Netzwerk umfasst mittlerweile ca 6500 Knoten und eine kumulierte Bandbreite von ca 6\,GBps\footnote{\url{http://torstatus.blutmagie.de/\#Stats}, Abgerufen am 18.12.2014}.

In einer groben Zusammenfassung über die Funktionsweise von Tor, kann man sagen, dass der 

\subsubsection{Ziele des Tor Netzwerks}
\label{sec:goals}

\subsubsection{Das Bedrohungs Modell}
\label{sec:threatmodel}

\subsubsection{Das Design des Tor Netzwerks}
\label{sec:tordesign}


- Source Routing\\

\subsubsection{Kommunikation - Cells, Circuits und Streams}
\label{sec:communication}

\subsubsection{Bandbreitenlimitierung und Staukontrolle}
\label{sec:communication}

\paragraph{Cross Circuit Interference Problem}
\label{sec:crosscircuitinterference}



- Cirtcuit construction\\
- Diffie Hellman\\
- SOCKS Proxy\\
- Cross Circuit Interference Problem (Unfair TCP Congestion Control)\\

%------------------------------------------------------------------------------
\subsection{IPSEC}
\label{sec:ipsec}

Basics zu IPSec\\
Erklärungen zu den Subprotokollen:\\
- Authentication Header (AH)\\
- Encapsulating Security Payload (ESP)\\
Erklärungen zu Operationmodes:\\
- transport mode\\
- tunnel mode\\

%==============================================================================
\section{Verwandte Arbeiten}
\label{sec:realtedwork}

-Einleitung in TCP-over-DTLS, der Ansatz den die Autoren versuchen zu verbessern. \\
-Aufzeigen der Probleme und der Punkte an denen die Verbesserungen ansetzen.\\

%==============================================================================
\section{PCTCP}
\label{sec:pctcp}



%------------------------------------------------------------------------------
\subsection{Kernel-mode per-circuit TCP}
\label{sec:kernelmode}

- Konzept der Verbindung innerhalb des netzwerks\\
- Illustrationen\\
- Vorteile des Deployments (Funktion des heterogenen Netzwerks (Plain tor + PCTCP)\\
- änderung am Verbinungsaufbau\\
- Resultierende Probleme\\

%------------------------------------------------------------------------------
\subsection{IPSec in PCTCP}
\label{sec:ipsecinpctcp}

- Läsung der Probleme mit IPSec\\
- Alternative Läsungen

%==============================================================================
\section{Evaluation}
\label{sec:evaluation}



%==============================================================================
\section{Fazit}
\label{sec:conclusion}



%==============================================================================
\section{Schriftliche Zusammenfassung}
\label{sec:zusam}

Die schriftliche Zusammenfassung soll sich im Aufbau an die Struktur eines kurzen wissenschaftlichen Beitrags (Konferenz-Paper) orientieren. Sie beginnt mit einem Titel, dem Namen des Verfassers sowie einer Zusammenfassung, welche die zentralen Ergebnisse des dargestellten Artikels prägnant darstellt. Der erste Abschnitt ist eine Einleitung, welche das Thema des Artikels motivieren und das Interesse des Lesers fär das kommende Material wecken soll.

Der Haupttext ist im Regelfall in weitere nummerierte Abschnitte zu unterteilen. Es besteht eine harte 8-Seitenbegrenzung (siehe unten).

Der Text muss mit einer Schlussbemerkungen enden, welche ebenfalls kurz (1--2 Absätze) die Beiträge des Artikels rekapituliert und bei Bedarf einen Ausblick gibt. Nachfolgend wird die Literaturliste platziert.

Es ist \LaTeX{} und der LNCS-Stil (\texttt{llncs}) zu verwenden. Fär die Literaturliste ist der Springer-Bibliographiestil \texttt{splncs} zu verwenden. Dieses Dokument stellt ein Beispiel fär die Verwendung des LNCS-Stils sowie des Springer-Bibliographiestils dar. Die 8-Seitenbegrenzung gilt fär das komplette Dokument inklusive des Titels und der Literaturliste. Es sind keine weiteren Elemente (Inhalts- und Abbildungsverzeichnisse, Danksagungen etc.) zu verwenden. Eine äbersicht der Literaturfelder fär das BibTeX kann der Wikipediaseite entnommen werden \cite{bibtex-wiki}.

\textbf{Die endgältige Ausarbeitung ist im PDF-Format abzugeben. Ebenfalls sollen alle Sourcen (.tex, Bilder, Bibliographie, etc) in einer kompilierbaren Form als Archiv (.zip, .tar.gz) abgegeben werden. Die 8-Sei\-ten\-be\-gren\-zung ist unbedingt einzuhalten! äberschreitungen der Seitenbegrenzung fähren zum Nichtbestehen des Proseminars. Modifikationen am LNCS-Stil sind nicht erlaubt.}

%==============================================================================
\section{Folien}
\label{sec:folien}

Die Folien sind Begleitmaterial fär den Vortrag und sollen nicht zwingend zum selbständigen Studium geeignet sein. Die Folien mässen ebenfalls auf Deutsch im PDF-Format eingereicht werden. Während ästhetische Aspekte nicht im Vordergrund stehen, sollten die Folien zumindest so kontrastreich sein, dass eine problemfreie visuelle Rezeption auch bei schwierigen Lichtverhältnissen gewährleistet ist.

Der Vortrag sollte mit einer kurzen (1 -- 2 Folien) Motivation des angegangenen Problems beginnen. Danach sollte eine Gliederungsfolie kommen, die je nach Vortragsstil später wiederholt gezeigt werden kann. Navigationselemente, wie sie etwa von Latex-Beamer-Style erzeugt werden, sollten nur eingesetzt werden, wenn eine sinnvolle Verwendung fär sie besteht (z.B.~häufiges Springen zu vorherigen Folien).

Die Hauptaufgabe des Vortrags besteht darin, den Zuhärer/innen das Gefähl fär das Thema, die Fragestellungen, das Erreichte und die offenen Fragen zu vermitteln. Formale Herleitungen sind spärlich einzusetzen und nach Mäglichkeit zu visualisieren (ohne dabei jedoch die Korrektheit des geschriebenen zu opfern). Längere Sätze (äber 2 Zeilen) sollten nach Mäglichkeit vermieden werden. Idealerweise sollten bei Aufzählungen kurze, prägnante unvollständige Sätze zum Einsatz kommen. Der Vortrag soll nicht darin bestehen, den Inhalt der Folien vorzulesen!

Am Ende der Präsentation sollte die Zusammenfassung des Vortrags (1 -- 2 Folien) gegeben werden. Eine Schlussfolie (\glqq Noch Fragen?\grqq, \glqq That's all folks!\grqq, Bilder von Haustieren und ähnliches) stellt erfahrungsgemää eine schlechtere Diskussionsgrundlage als eine Zusammenfassungsfolie dar.

%==============================================================================
\nocite{rfc:socks}
\nocite{rfc:ipsec}
\nocite{rfc:tls}
\nocite{tor}
\nocite{pctcp}
\nocite{tcp-over-dtls}
\bibliographystyle{splncs}
\bibliography{literatur}
%==============================================================================
\end{document}
