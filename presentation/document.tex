\documentclass{beamer}
\usepackage[utf8]{inputenc}
\usepackage[ngerman]{babel} % neue deutsche Rechtschreibung 
\usepackage{babelbib}       % deutsches Literaturverzeichnis
\usepackage{verbatim}

% Background
\usepackage{tikz}
\usetikzlibrary{fadings}
\setbeamertemplate{background}{
\begin{tikzpicture}
\path (0,0) rectangle (\paperwidth,\paperheight);
\node[scope fading=west,inner sep=0pt,outer sep=0pt,anchor=north east] at(\paperwidth*1.3,\paperheight*0.83) {\includegraphics[height=\paperheight]{pics/uni}};
\end{tikzpicture}
}

% \usepackage{moreverb}
% \usepackage{array}
% \usepackage{subfigure}
% \usepackage{graphicx}
%\usepackage{amsmath}
%\usepackage{amsfonts}
%\usepackage{amssymb}
%\usepackage{cite}
%\usepackage{url}

%===============================================================================
%///////////////////////////////////////////////////////////////////////////////
%vvvvvvvvvvvvvvvvvvvvvvvvvvvvvvvvvvvvvvvvvvvvvvvvvvvvvvvvvvvvvvvvvvvvvvvvvvvvvvv
\usetheme{Rochester}
\usecolortheme{seagull}
% Remove navigation
\beamertemplatenavigationsymbolsempty
% \setcounter{tocdepth}{2}
% \usecaptiontemplate{
% \scriptsize
% %\structure{\insertcaptionname:}
% \insertcaption
% }
% \setcounter{tocdepth}{2}

% Use bullet as item for lists
\setbeamertemplate{itemize items}{$\bullet$}

% %\setbeamertemplate{itemize subitem}{{\tiny $\spadesuit$}}
% %\setbeamertemplate{itemize subsubitem}{{\tiny $\heartsuit$}}
% \setbeamertemplate{footline}[frame number]
% \setbeamertemplate{sections/subsections in toc}[circle]{}
%^^^^^^^^^^^^^^^^^^^^^^^^^^^^^^^^^^^^^^^^^^^^^^^^^^^^^^^^^^^^^^^^^^^^^^^^^^^^^^^
%///////////////////////////////////////////////////////////////////////////////
%///////////////////////////////////////////////////////////////////////////////
%///////////////////////////////////////////////////////////////////////////////
\begin{document}%///////////////////////////////////////////////////////////////
%///////////////////////////////////////////////////////////////////////////////
%///////////////////////////////////////////////////////////////////////////////
%///////////////////////////////////////////////////////////////////////////////
%===============================================================================
\title{Per-Circuit TCP-over-IPsec Transport\\for Anonymous Communication Overlay Networks}   
\author{Manuel Schneider}
\institute{Albert Ludwigs Universität - Institut für Informatik} 
\date{\today} 
%------------------------------------------------------------------------------

\begin{frame}[plain]
\titlepage
\end{frame}

\begin{frame}{Motivation}
\end{frame}

%===============================================================================
%///////////////////////////////////////////////////////////////////////////////
\section{Grundlagen}
%///////////////////////////////////////////////////////////////////////////////
%===============================================================================

\begin{frame}{Übersicht}
\tableofcontents[sectionstyle=show/shaded, subsectionstyle=show/hide/hide]
\end{frame}

%------------------------------------------------------------------------------
\subsection{Tor}

\begin{frame}{\subsecname}{\secname}
Basics zu Tor (Ausmaß in der Präsentation abhäangig von Dirk).
Reduziert sich minimal auf die Performancemängel und die Transportlayergeschich-
te, die für diese Arbeit besonders von Belang ist.
Themen:
\end{frame}

%------------------------------------------------------------------------------
\subsection{IPSec}
\begin{frame}{\subsecname}{\secname}
- Ipsec generell (Was warum wie wo)\\
\end{frame}

\begin{frame}{Subprotokolle}{\secname/\subsecname}
- Authentication Header (AH)\\
- Encapsulating Security Payload (ESP)\\
\end{frame}

\begin{frame}{Operationmodi}{\secname/\subsecname}
- transport mode\\
- tunnel mode\\
\end{frame}

%===============================================================================
%///////////////////////////////////////////////////////////////////////////////
\section{Verwandte Arbeiten}
%///////////////////////////////////////////////////////////////////////////////
%===============================================================================

\begin{frame}{Übersicht}
\tableofcontents[sectionstyle=show/shaded, subsectionstyle=show/hide/hide]
\end{frame}

\begin{frame}{TCP-over-DTLS}{\secname}
-Einleitung in TCP-over-DTLS\\
-Wo wird es verwendet0\\

\end{frame}

\begin{frame}{TCP-over-DTLS}{\secname}
-Funktionsweise

\end{frame}

\begin{frame}{Probleme beim TCP-over-DTLS}{\secname}
-Aufzeigen der Probleme und der Punkte an denen die Verbesserungen ansetzen.\\
-Einleitung in das PCTCP
\end{frame}

%===============================================================================
%///////////////////////////////////////////////////////////////////////////////
\section{PCTCP}
%///////////////////////////////////////////////////////////////////////////////
%===============================================================================

\begin{frame}{Übersicht}
\tableofcontents[sectionstyle=show/shaded, subsectionstyle=show/hide/hide]
\end{frame}

\begin{frame}{\secname}{\subsecname}
-Einleitung in PCTCP. Wo wird es eingesetzt? Anlehnung an DTLS
\end{frame}

%------------------------------------------------------------------------------
\subsection{Kernel-mode per-circuit TCP}

\begin{frame}{Konzept}{\secname/\subsecname}
- Konzept der Verbindung innerhalb des netzwerks\\
- Schön mit Illustration
\end{frame}

\begin{frame}{Notwendige Änderungen}{\secname/\subsecname}
- Änderung am Verbinungsaufbau\\
\end{frame}

\begin{frame}{Deployment}{\secname/\subsecname}
- Vorteile des Deployments (Funktion des heterogenen Netzwerks (Plain tor + PCTCP)\\
\end{frame}

\begin{frame}{Probleme}{\secname/\subsecname}
- Resultierende Probleme\\
\end{frame}

%------------------------------------------------------------------------------
\subsection{IPSec in PCTCP}

\begin{frame}{}{\secname/\subsecname}
- Lösung der Probleme mit IPSec\\
\end{frame}

\begin{frame}{}{\secname/\subsecname}
- Alternative Lösungen
\end{frame}

%===============================================================================
%///////////////////////////////////////////////////////////////////////////////
\section{Evaluation}
%///////////////////////////////////////////////////////////////////////////////
%===============================================================================

\begin{frame}{Übersicht}
\tableofcontents[sectionstyle=show/shaded, subsectionstyle=show/hide/hide]
\end{frame}

%------------------------------------------------------------------------------
\begin{frame}{\secname}{\subsecname}
\end{frame}



%===============================================================================
%///////////////////////////////////////////////////////////////////////////////
%===============================================================================

\begin{frame}[plain]{Übersicht}
	\tableofcontents[sectionstyle=show/show, subsectionstyle=show/hide/hide]
\begin{center}
Vielen Dank für die Aufmerksamkeit!
\end{center}
\end{frame}

%===============================================================================
%///////////////////////////////////////////////////////////////////////////////
%///////////////////////////////////////////////////////////////////////////////
%///////////////////////////////////////////////////////////////////////////////
%///////////////////////////////////////////////////////////////////////////////
%===============================================================================

\nocite{rfc:socks}
\nocite{rfc:ipsec}
\nocite{rfc:tls}
\nocite{pctcp}
\nocite{tcp-over-dtls}
{\tiny
\bibliographystyle{plain}
\bibliography{bibliography}
}

%===============================================================================
%///////////////////////////////////////////////////////////////////////////////
%///////////////////////////////////////////////////////////////////////////////
%///////////////////////////////////////////////////////////////////////////////
%///////////////////////////////////////////////////////////////////////////////
\end{document}%/////////////////////////////////////////////////////////////////
%///////////////////////////////////////////////////////////////////////////////
%///////////////////////////////////////////////////////////////////////////////
%///////////////////////////////////////////////////////////////////////////////
%///////////////////////////////////////////////////////////////////////////////
%==============================================/=================================

